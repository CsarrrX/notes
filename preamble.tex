% =========================
%      PREAMBLE.TEX
% =========================

\usepackage[utf8]{inputenc}
\usepackage[T1]{fontenc}
\usepackage[spanish,es-nodecimaldot,mexico]{babel} % Español mx, mantiene el punto decimal
\usepackage[a4paper, top=2.5cm, bottom=2.5cm, left=2.5cm, right=2.5cm]{geometry}

% --- PAQUETES MATEMÁTICOS ---
\usepackage{amsmath}
\usepackage{amssymb}
\usepackage{amsthm}     
\usepackage{mathtools}
\usepackage{cleveref}

% --- GRÁFICOS (INKSCAPE SETUP) ---
\usepackage{graphicx}
\usepackage{enumitem}
\usepackage{import}
\usepackage{pdfpages}
\usepackage{transparent}
\usepackage{microtype}
\usepackage{xcolor}
\usepackage{float}

% Comando para importar figuras de Inkscape
\newcommand{\incfig}[1]{%
    \def\svgwidth{\columnwidth}
    \import{./figures/}{#1.pdf_tex}
}

\pdfsuppresswarningpagegroup=1

% --- HEADERS Y FOOTERS (Estilo imagen) ---
\usepackage{fancyhdr}
\pagestyle{fancy}
\fancyhf{} % Limpiar todo

% Configuración del Header
\setlength{\headheight}{15pt}
\renewcommand{\headrulewidth}{0.4pt} 
\fancyhead[R]{\nouppercase{\leftmark}} 
\fancyhead[L]{} 

% Footer: Número de página al centro
\fancyfoot[C]{\thepage}

% --- CONFIGURACIÓN DE CAJAS (mdframed) ---
\usepackage[framemethod=TikZ]{mdframed}

% Estilo para Definición, Teorema, etc. (Caja negra completa)
\mdfdefinestyle{theobox}{
  linewidth=0.6pt,
  linecolor=black,
  roundcorner=0pt,
  innertopmargin=0pt,    % Espacio interno superior para evitar colapso
  innerbottommargin=6pt,
  innerleftmargin=10pt,
  innerrightmargin=10pt,
  skipabove=12pt,
  skipbelow=10pt,
  nobreak=true           
}

% Estilo para Demostración (Línea lateral fina)
\mdfdefinestyle{proofline}{
  linewidth=1pt,
  leftline=true,
  rightline=false,
  topline=false,
  bottomline=false,
  linecolor=black,
  innertopmargin=6pt,
  innerbottommargin=6pt,
  innerleftmargin=10pt,
  innerrightmargin=0pt,
  skipabove=8pt,
  skipbelow=8pt
}

% --- TEOREMAS (Estilo Clásico) ---
\theoremstyle{plain} 
\newtheorem{teorema}{Teorema}[section]
\newtheorem{proposicion}[teorema]{Proposición}
\newtheorem{lema}[teorema]{Lema}

\theoremstyle{definition}
\newtheorem{definicion}[teorema]{Definición}
\newtheorem{axioma}{Axioma}[section] 

\theoremstyle{remark} 
\newtheorem*{nota}{Nota}
\newtheorem*{ejemplo}{Ejemplo}

% Aplicar cajas automáticas
\surroundwithmdframed[style=theobox]{teorema}
\surroundwithmdframed[style=theobox]{proposicion}
\surroundwithmdframed[style=theobox]{lema}
\surroundwithmdframed[style=theobox]{definicion}
\surroundwithmdframed[style=theobox]{axioma}

% --- REDEFINIR DEMOSTRACIÓN (Línea lateral negra) ---
\renewenvironment{proof}[1][\proofname]{%
  \begin{mdframed}[style=proofline]%
  \noindent\textit{#1. }%
}{%
  \hfill$\square$ 
  \end{mdframed}%
}

% --- COMANDO PARA LA PORTADA ---
\newcommand{\crearportada}{
    \begin{titlepage}
        \centering
        \vspace*{2cm}
        {\Huge \textbf{\materia} \par}
        \vspace{1cm}
        {\Large \textbf{Clave:} \clave \par}
        \vspace{0.5cm}
        {\Large \semestre \par}
        \vspace{4cm}
        \textbf{\Large \usuario}
        \vfill
        {\large \today \par}
    \end{titlepage}
}
