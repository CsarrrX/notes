% preamble.tex
\usepackage[utf8]{inputenc}
\usepackage[T1]{fontenc}
\usepackage[spanish,es-nodecimaldot,mexico]{babel} % Español mx, mantiene el punto decimal
\usepackage[a4paper, top=2.5cm, bottom=2.5cm, left=2.5cm, right=2.5cm]{geometry}

% --- PAQUETES MATEMÁTICOS ---
\usepackage{amsmath}
\usepackage{amssymb}
\usepackage{amsthm}     % Para teoremas sin cajitas
\usepackage{mathtools}

% --- GRÁFICOS (INKSCAPE SETUP) ---
\usepackage{graphicx}
\usepackage{import}
\usepackage{pdfpages}
\usepackage{transparent}
\usepackage{xcolor}
\usepackage{float}

% Comando para importar figuras de Inkscape
\newcommand{\incfig}[1]{%
    \def\svgwidth{\columnwidth}
    \import{./figures/}{#1.pdf_tex}
}

\pdfsuppresswarningpagegroup=1

% --- HEADERS Y FOOTERS (Estilo imagen) ---
\usepackage{fancyhdr}
\pagestyle{fancy}
\fancyhf{} % Limpiar todo

% Configuración del Header: Línea abajo y Sección a la derecha
\setlength{\headheight}{15pt}
\renewcommand{\headrulewidth}{0.4pt} % Grosor de la línea
\fancyhead[R]{\nouppercase{\leftmark}} % Nombre de la sección (Clase) a la derecha
\fancyhead[L]{} % Izquierda vacía

% Footer: Número de página al centro
\fancyfoot[C]{\thepage}

% --- TEOREMAS (Estilo Clásico) ---
% 1. Estilo para lo que lleva itálica en el cuerpo (Teoremas, Prop, Lemas)
\theoremstyle{plain} 
\newtheorem{teorema}{Teorema}[section]
\newtheorem{proposicion}[teorema]{Proposición}
\newtheorem{lema}[teorema]{Lema}

% 2. Estilo para lo que lleva texto normal (Definiciones y Axiomas)
\theoremstyle{definition}
\newtheorem{definicion}[teorema]{Definición}
\newtheorem{axioma}{Axioma}[section] % Numeración por sección independiente

% 3. Estilo para lo que lleva título en itálica y texto normal (Notas y Ejemplos)
\theoremstyle{remark} 
\newtheorem*{nota}{Nota}
\newtheorem*{ejemplo}{Ejemplo}

% --- COMANDO PARA LA PORTADA ---
\newcommand{\crearportada}{
    \begin{titlepage}
        \centering
        \vspace*{2cm}
        
        {\Huge \textbf{\materia} \par}
        \vspace{1cm}
        {\Large \textbf{Clave:} \clave \par}
        \vspace{0.5cm}
        {\Large \semestre \par}
        
        \vspace{4cm}
        \textbf{\Large \usuario}
        
        \vfill
        {\large \today \par}
    \end{titlepage}
}
