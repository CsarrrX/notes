\chapter{Lógica}

Esta sección se basa en gran medida en el \textbf{Apéndice A} del libro \textit{Analysis I} de Terence Tao. Su objetivo es introducir los fundamentos de la \textbf{lógica matemática}, entendida como el lenguaje formal de las demostraciones. El estudio de esta es indispensable, ya que en matemáticas nos interesa establecer verdades absolutas. Si bien existen otros sistemas lógicos que consideran nociones de verdad no absoluta, estos no son relevantes para los fines de las presentes notas.

\section{Enunciados matemáticos}

Todo argumento matemático está formado por \textit{enunciados matemáticos}. Estos son enunciados precisos que hacen referencia a distintos \textit{objetos matemáticos} (como números, vectores o funciones) y a las \textit{relaciones u operaciones} definidas entre ellos, (tales como la suma, la integración, la igualdad o la composición). Dichos objetos pueden ser constantes o \textit{variables}. Un aspecto fundamental es que todo enunciado matemático posee un valor de verdad bien definido: \textbf{es verdadero o falso}, y no admite ambos valores simultáneamente.

\begin{ejemplo}\label{ej1}
    $2 + 2 = 4$ es un enunciado \textit{verdadero} y $2 + 2 = 5$ es un enunciado \textit{falso}.
\end{ejemplo}

Cabe notar que no toda combinación de símbolos matemáticos es un enunciado matemático, por ejemplo la combinación: $= 2 + + 4 = + = 1$ \textbf{no} es un enunciado matemático y se le llama \textit{mal-formado} o \textit{mal-definido}. Este tipo de enunciados no tienen ningún valor de verdad.

\begin{ejemplo}\label{ej2}
    $\frac{0}{0} = 1$ es un ejemplo un poco más sutil de un enunciado mal-definido, esto ya que la división sobre 0 no está definida.
\end{ejemplo}

Un argumento matemático no debe contener ningún enunciado mal formado, entonces, si por ejemplo tenemos la expresión $\frac{x}{y} = z$ tenemos que asegurarnos $y \neq 0$.
Es importante señalar que, en la lógica matemática, la veracidad o falsedad de un enunciado es una propiedad inherente del mismo y no depende de la opinión de quien lo enuncia o interpreta. Esta característica junto con el valor de verdad bien definido de los enunciados fundamenta un método clásico de demostración: para establecer la veracidad de un enunciado, puede suponerse su falsedad y, a partir de dicha suposición, derivar una contradicción lógica (el método se abarca más adelante).

\begin{nota}
    Básicamente, para demostrar que un enunciado es verdadero, basta con demostrar que no es falso, y de la misma forma, para demostrar que falso, basta con demostrar que no es verdadero.
\end{nota}

Ahora, que un enunciado sea verdadero, no significa que sea útil o eficiente. Por ejemplo el enunciado $2 = 2$ es verdadero, pero esta lejos de ser algo útil. O el enunciado $4 \leq 4$ también es verdadero, pero no muy eficiente ($4 = 4$ es más preciso). Y viceversa, un enunciado puede ser falso pero muy útil, por ejemplo $\pi = 3.1416$ es falso, sin embargo es muy útil como una \textit{aproximación}. 

En razonamiento matemático, solo nos importa la \textbf{verdad}, no tanto la utilidad ni eficiencia; esto ya que la verdad es objetiva, podemos deducir enunciados verdaderos de reglas precisas, mientras que la utilidad y eficiencia subjetivas, dependen de la opinión del observador.

Los enunciados son diferentes de las \textit{expresiones}. Como ya vimos los enunciados son verdaderos o falsos, mientras que las expresiones son \textit{secuencias} de caracteres matemáticos que producen un objeto matemático, por ejemplo, $2 \cdot 3 + 5$ es una expresión cuyo resultado es un número. Mientras que $ 2 \cdot 3 + 5 = 9$ es un enunciado, no una expresión. Como con los enunciados, las expresiones pueden ser bien o mal definidas, $2 + \frac{3}{0}$ es una expresión mal-definida. Ejemplos más sutiles de expresiones así ocurren, por ejemplo, cuando intentamos multiplicar matrices con dimensiones mal definidas, evaluamos funciones fuera de su dominio, etc. 
Podemos crear enunciados con expresiones al hacer uso de \textit{relaciones} (como $=, <, \in$, etc.) o propiedades (como: <<es primo>>, <<es continua>>, <<es invertible>>, etc.), por ejemplo, <<$3 + 2$ es un número primo>> o $3 + 28 < 40$ son enunciados matemáticos.
También podemos crear una \textit{proposición compuesta} usando enunciados más primitivos al \textit{conectarlos lógicamente}. 

\section{Conectores lógicos} 

En lógica matemática, los conectores lógicos constituyen las herramientas fundamentales para construir y analizar proposiciones compuestas a partir de enunciados simples. Mediante estos operadores es posible formalizar relaciones como la conjunción, la disyunción, la implicación y la negación, permitiendo expresar razonamientos de manera precisa y rigurosa. A continuación se presentan los más utilizados. 

\begin{nota}
    Aunque en esencia un enunciado y una proposición son lo mismo, en estas notas nos referiremos a \textit{proposiciones} como enunciados un poco más complejos, sucesiones de enunciados más primitivos conectados lógicamente.
\end{nota}

\begin{definicion}[Conjunción ($\land$)]
    Si $p$ es un enunciado y $q$ es un enunciado, entonces la proposición <<$p$ y $q$>> es verdadera si tanto $p$ como $q$ son verdaderos y falsa de otra forma. Por ejemplo, la proposición $2 + 2 = 4$ y $3 + 3 = 9$ es falsa, mientras que $2 + 2 = 4$ y $2 + 2 = 4$ es verdadera (aunque redundante).
\end{definicion}

\begin{definicion}[Disyunción ($\lor$)]
    Si $p$ es un enunciado y $q$ es un enunciado, entonces la proposición <<$p$ o $q$>> es verdadera si uno de los dos es verdadero (o ambos) y falsa solo cuando ambos son falsos. Por ejemplo, la proposición $2 + 2 = 1$ o $3 + 2 = 0$ es falsa, mientras que $2 = 2$ o $0 = 5$ es verdadera. 
\end{definicion}

\begin{nota}
    También existe el <<o exclusivo>>, en este solo nos interesa que UNO de los enunciados sea verdadero, si ambos son verdaderos entonces la proposición es falsa. 
\end{nota}

\begin{definicion}[Negación ($\neg$)]
    Si $p$ es un enunciado, la proposición <<$p$ no es verdadero>> es verdadera si y solo si $p$ es falso, y es falsa si y solo si $p$ es verdadero. La negación se presenta bastante en matemáticas (por ejemplo para demostrar por contradicción), gracias a esto ahora se presentan las reglas de la negación y ejemplos de negación de proposiciones.   
\end{definicion}

Primero, la negación convierte la conjunción en disyunción y viceversa, por ejemplo, la proposición <<($p$ y $q$) es falsa>> se convierte en <<$p$ es falsa o $q$ es falsa>> (nótese que también invierte el valor de verdad de los enunciados), y de la misma forma si tenemos <<($p$ o $q$) es falsa>> se convierte en <<$p$ es falsa y $q$ es falsa>>. Otra regla es que la doble negación se cancela, es decir, si tenemos un enunciado $p$ y la proposición <<$p$ es falso>>, la negación de esta proposición es <<$p$ es verdadero>>. 

\begin{ejemplo}
    Un ejemplo común de la negación se da en intervalos, por ejemplo, para negar $1 < x < 2$ tenemos que partir la proposición en $1 < x$ y $2 > x$ ahora, aplicando la regla para negarlo tenemos $1 \geq x$ o $2 \leq x$.
\end{ejemplo}

El dominio de la negación de proposiciones se da practicándola, no es algo particularmente complejo, sin embargo más adelante presentaremos cuantificadores y variables, los cuales conllevarán nuevas reglas de negación, entonces es importante recordar estas reglas básicas (también conocidas como las leyes de De Morgan).

\begin{definicion}[Si y solo si ($\iff$)] 
    Si $p$ es un enunciado y $q$ es un enunciado, decimos <<$p$ es verdadero si y solo si $q$ es verdadero>> cuando son \textit{lógicamente equivalentes}, es decir, si $p$ es verdadero $q$ es verdadero, y si $p$ es falso entonces $q$ es falso, y de la misma forma, tenemos que si $q$ es verdadero, $p$ es verdadero y si $q$ es falso, entonces $p$ es falso.   
\end{definicion}


\subsection{Implicación}

Hemos llegado al conector lógico menos intuitivo, la \textbf{implicación} ($\implies$). Si $p$ y $q$ son enunciados, entonces la proposición <<si $p$ entonces $q$>> es la implicación de $p$ a $q$. El significado de la proposición <<si $p$, entonces $q$>> depende de si $p$ es verdadero o falso. Si $p$ es verdadero, entonces <<si $p$, entonces $q$>> es verdadero cuando $q$ es verdadero, y falso cuando $q$ es falso. Si, por el contrario, $p$ es falso, entonces <<si $p$, entonces $q$>> es siempre verdadero, independientemente de si $q$ es verdadero o falso. Dicho de otra manera, cuando $p$ es verdadero, la proposición <<si $p$, entonces $q$>> implica que $q$ es verdadero. Pero cuando $p$ es falso, la proposición <<si $p$, entonces $q$>> no ofrece información sobre si $q$ es verdadero o no; la proposición es verdadera, pero vacua (es decir, no transmite ninguna información nueva más allá del hecho de que la hipótesis es falsa).

\begin{nota}
    Una forma más intuitiva de entenderlo es con un ejemplo. Sean $p$: <<llueve>> y $q$: <<el suelo está mojado>>. La implicación <<si $p$, entonces $q$>> funciona como una regla o promesa:
    \begin{itemize}[noitemsep]
        \item Si llueve ($p$ es V) y el suelo está mojado ($q$ es V), la regla se cumple: la implicación es \textbf{verdadera}.
        \item Si llueve ($p$ es V) pero el suelo está seco ($q$ es F), la regla se ha roto: la implicación es \textbf{falsa}.
        \item Si no llueve ($p$ es F), la regla no dice nada sobre el suelo. No importa si está mojado o seco, la afirmación original no ha sido contradicha, por lo que la implicación se considera \textbf{verdadera} por defecto.
    \end{itemize}
\end{nota}

Para demostrar una implicación, la manera usual de hacerlo es primero asumir que la hipótesis es verdadera y usar esto (junto con otros hechos e hipótesis verdaderas que conozcas) para deducir la conclusión. Este procedimiento es perfectamente válido, incluso si la hipótesis resulta falsa después; la implicación no garantiza nada acerca de la veracidad de la hipótesis, y solo garantiza la veracidad de la conclusión dado condicionalmente que la hipótesis sea verdadera. Por ejemplo, la demostración de la siguiente proposición es verdadera aunque tanto la hipótesis como la conclusión son falsas:

\begin{proposicion}
    Si $2 + 2 = 5$ entonces $4 = 10 - 4$
\end{proposicion}

\begin{proof}
    Asumimos $2 + 2 = 5$. Multiplicamos ambos lados por 2, obtenemos $4 + 4 = 10$. Restamos $4$ de ambos lados, y obtenemos $4 = 10 - 4$ como queríamos.
\end{proof}

Un error común al demostrar una implicación es asumir la conclusión para llegar a la hipótesis. Esto se debe a que si $p$ implica $q$, no necesariamente $q$ implica $p$, Un ejemplo de una demostración incorrecta sería:

\begin{proposicion}
    Supongamos que $2x + 3 = 7$. Demuestra que $x = 2$
\end{proposicion}

\begin{proof}
    (Incorrecta) $x = 2$; entonces $2x = 4$; entonces $2x + 3 = 7$
\end{proof}

\textbf{Recíproca:} La recíproca de una implicación $p \implies q$ es la proposición $q \implies p$. Es fundamental entender que la veracidad de una implicación no garantiza la veracidad de su recíproca. De hecho, confundir ambas es uno de los errores más comunes en las demostraciones. Cuando tanto la implicación ($p \implies q$) como su recíproca ($q \implies p$) son verdaderas, decimos que los enunciados son lógicamente equivalentes, lo cual se denota con el conector de doble implicación ($p \iff q$). 

\textbf{Contrapositiva:} La contrapositiva es una equivalencia lógica fundamental que establece que una implicación es lógicamente igual a la negación de sus términos en orden inverso, expresándose formalmente como $(p \implies q) \equiv (\neg q \implies \neg p)$. En términos de razonamiento, esto significa que si la veracidad de un antecedente garantiza un consecuente, la ausencia de dicho consecuente implica necesariamente la ausencia del antecedente original.

\begin{nota}
    La contrapositiva es una forma equivalente de expresar el ejemplo anterior: <<si el suelo no está mojado, entonces no llueve>> ($\neg q \implies \neg p$). Podemos entender su validez analizando los mismos escenarios:
    \begin{itemize}[noitemsep]
        \item Si el suelo está seco ($\neg q$ es V) y efectivamente no llueve ($\neg p$ es V), la regla se cumple: la implicación es \textbf{verdadera}.
        \item Si el suelo está seco ($\neg q$ es V) pero resulta que está lloviendo ($\neg p$ es F), la regla se ha roto: la implicación es \textbf{falsa}.
        \item Si el suelo está mojado ($\neg q$ es F), la regla no impone ninguna condición. No importa si llueve o no, la afirmación no puede ser desmentida porque solo hablaba del caso en que el suelo estuviera seco. Por tanto, la implicación se considera \textbf{verdadera}.
    \end{itemize}
\end{nota}

\textbf{Reducción al absurdo:} Una aplicación directa de la implicación es que si $p$ implica algo que sabemos que es falso, entonces $p$ mismo debe ser falso. Este es el principio de la \textit{reductio ad absurdum}: para demostrar que una afirmación es falsa, asumimos primero que es verdadera y mostramos que esto conduce a una contradicción lógica (un enunciado que es simultáneamente verdadero y falso). Por ejemplo:

\begin{proposicion}
    No existe un número entero mayor que los demás.
\end{proposicion}

\begin{proof}
    \textbf{Asumimos lo contrario:} supongamos que existe un número entero $N$ mayor que todos los demás. \textbf{Derivamos una consecuencia y un absurdo (contradicción):} por propiedades numéricas sabemos que $N + 1$ es un número mayor y es entero, entonces llegamos a que $N$ es el entero más grande y al mismo tiempo no es el entero más grande. Por lo tanto no puede existir un número entero más grande que todos los demás. 
\end{proof}

\section{La estructura de las demostraciones}

Para demostrar una proposición, uno normalmente empieza asumiendo la hipótesis y avanzando hacia la conclusión; esta es el enfoque \textit{directo} para demostrar una proposición. Por ejemplo:

\begin{proposicion}\label{prop1}
    $A$ implica $B$
\end{proposicion}

\begin{proof}
    Asumimos $A$ es verdadero. Como $A$ es verdadero, $C$ es verdadero. Como $C$ es verdadero, $D$ es verdadero. Como $D$ es verdadero, $B$ es verdadero, como queríamos. 
\end{proof}

Pero esta no es la única forma de demostrar esto, también podemos trabajar desde la conclusión (nunca asumiéndola como verdadera) y viendo que se necesitaría para que se cumpliera la implicación. Por ejemplo una demostración típica de \cref{prop1} se vería así:

\begin{proof}
    Para demostrar $B$, bastaría con demostrar $D$. Como $C$ implica $D$, solo tenemos que demostrar $C$. Pero $C$ sigue de $A$. 
\end{proof}

O también podemos aplicar una combinación de ambos enfoques de demostración, por ejemplo otra demostración válida de \cref{prop1} sería:

\begin{proof}
    Para demostrar $B$, bastaría con demostrar $D$. Entonces ahora demostremos $D$. Como tenemos $A$ por hipótesis, tenemos $C$. Como $C$ implica $D$, tenemos $D$ como queríamos.
\end{proof}

Las demostraciones pasadas fueron muy sencillas y simples de entender ya que solo hay una hipótesis y una conclusión. Sin embargo cuando hay múltiples hipótesis y conclusiones la demostración se divide en casos, entonces se pueden volver más complicadas. Por ejemplo: 

\begin{proposicion}
Suponga que $A$ y $B$ son verdaderas. Entonces $C$ y $D$ son verdaderas.
\end{proposicion}

\begin{proof}
Dado que $A$ es verdadera, $E$ es verdadera. A partir de $E$ y $B$ sabemos que $F$ es verdadera. Además, a la luz de $A$, para demostrar $D$ basta con mostrar $G$. Ahora existen dos casos: $H$ e $I$. Si $H$ es verdadera, entonces de $F$ y $H$ obtenemos $C$, y de $A$ y $H$ obtenemos $G$. Si, por el contrario, $I$ es verdadera, entonces de $I$ tenemos $G$, y de $I$ y $G$ obtenemos $C$. Por lo tanto, en ambos casos obtenemos tanto $C$ como $G$ y, por consiguiente, $C$ y $D$.
\end{proof}

Otra forma de demostrar una implicación es por \textit{contradicción} como se vio en \textbf{Implicación}, por ejemplo:

\begin{proposicion}
    Supón que $A$ es verdadero, entonces $B$ es verdadero
\end{proposicion}

\begin{proof}
    Supongamos por contradicción que $B$ es falso. Esto implicaría que $C$ es falso. Pero como $A$ es verdadero, implica que $D$ es falso, lo que contradice $C$. Por lo tanto $B$ debe ser verdadero.
\end{proof}

Como se puede ver hay muchos enfoques para resolver problemas, sin embargo en algunos contextos unos enfoques son más eficientes que otros, entonces, si notas que hay más de una forma para empezar un problema, puedes probar la que se vea más sencilla, pero tienes que estar preparado para cambiar de enfoque si empieza a parecer imposible. 

\section{Variables y cuantificadores}

Uno puede llegar muy lejos usando solamente enunciados primitivos (como $2 + 2 = 4$ o <<yo tengo el pelo blanco>>), conectándolos con lógica, y construyendo hipótesis y conclusiones a base de ello, a esto se le conoce como \textit{lógica booleana}. Sin embargo para la matemática este nivel de lógica no es suficiente, ya que no contiene el concepto fundamental de las \textit{variables}, una \textit{variable} es un símbolo, como $x$ o $n$ que denota un objeto matemático (un entero, vector, función, matriz, etc.) casi en todas las situaciones el \textit{tipo} del objeto debería de declararse, de otra forma, sería difícil formar enunciados bien-definidos a partir de ellos. Además, podemos formar expresiones y enunciados con estas variables, por ejemplo si $x$ es un número real, $x + 3$ es una expresión y $x + 3 = 5$ es un enunciado. Sin embargo, la veracidad del enunciado está ligada a la variable, en este caso tenemos que el enunciado es verdadero \textbf{si y solo si} $x = 2$. Una \textbf{variable libre} es un símbolo cuyo valor no está especificado (ej. $x+3=5$), por lo que la expresión carece de un valor de verdad definido. En contraste, una \textbf{variable ligada} ocurre cuando se fija su valor (ej. "Sea $x=2$") o se limita mediante cuantificadores, otorgando al enunciado un valor de verdad definido.

\begin{definicion}[Cuantificador Universal ($\forall$)]
    La proposición $\forall x \in T, P(x)$ significa que la propiedad $P(x)$ es verdadera para \textbf{todos} los elementos de tipo $T$. Para demostrar su validez, se debe argumentar sobre un $x$ arbitrario; para refutarla, basta con exhibir un solo contraejemplo.
\end{definicion}

\begin{definicion}[Cuantificador Existencial ($\exists$)]
    La proposición $\exists x \in T : P(x)$ significa que existe \textbf{al menos un} elemento en $T$ para el cual $P(x)$ es verdadero. Su demostración consiste en encontrar o construir un ejemplo específico que cumpla la propiedad.
\end{definicion}

La negación de estas proposiciones sigue reglas de intercambio fundamentales, conocidas como la dualidad de los cuantificadores:

\begin{equation*}
    \neg(\forall x, P(x)) \iff \exists x, \neg P(x)
\end{equation*}
\begin{equation*}
    \neg(\exists x, P(x)) \iff \forall x, \neg P(x)
\end{equation*}

\subsection{Cuantificadores Anidados}
El orden de los cuantificadores es fundamental para el significado de una proposición. Se distinguen dos casos principales:

\begin{proposicion}[Conmutatividad]
El orden es irrelevante si los cuantificadores son del mismo tipo; es decir, las siguientes equivalencias lógicas son válidas:
\[
\begin{aligned}
    \forall x \forall y \, P(x, y) &\equiv \forall y \forall x \, P(x, y) \\
    \exists x \exists y \, P(x, y) &\equiv \exists y \exists x \, P(x, y)
\end{aligned}
\]
\end{proposicion}
\begin{ejemplo}[Cuantificadores Mixtos]
Si se combinan $\forall$ y $\exists$, el orden define la \textbf{dependencia} de las variables. Sea el dominio $\mathbb{Z}$:
\begin{alignat*}{3}
    &\forall n \exists m (m > n) \quad && \text{\textbf{Verdadero.}} \quad && \text{Cada $n$ tiene su propio $m$ (ej. $m = n + 1$).} \\
    &\exists m \forall n (m > n) \quad && \text{\textbf{Falso.}}     \quad && \text{Requeriría un $m$ universalmente superior.}
\end{alignat*}
\end{ejemplo}

\begin{nota}
En la secuencia $\forall x \exists y$, el valor de $y$ se elige \textit{después} de conocer $x$. En $\exists y \forall x$, el valor de $y$ debe ser el mismo para todos los $x$.
\end{nota}

\section{Igualdad}

La igualdad es una relación entre dos objetos $x, y$ del mismo tipo $T$. Su veracidad depende de los valores y de la definición de igualdad para dicho tipo (ej. $12 = 2$ es verdadero en aritmética modular de módulo 10, pero falso en aritmética ordinaria). Para propósitos lógicos, la igualdad debe satisfacer cuatro axiomas fundamentales:

\begin{axioma}[Reflexivo]
    Para cualquier objeto $x$, se tiene $x = x$
\end{axioma}

\begin{axioma}[Simetría]
    Si $x = y$, entonces $y = x$ 
\end{axioma}

\begin{axioma}[Transitivo]
    Si $x = y$ y $y = z$ entonces $x = z$
\end{axioma}

\begin{axioma}[Sustitución]
    Si $x = y$, entonces $f(x) = f(y)$ para cualquier función $f$. Asimismo, si $P(x)$ es una propiedad, entonces $P(x) \iff P(y)$
\end{axioma}

Por ejemplo tenemos: 
\begin{itemize}[noitemsep]
    \item Si $x, y \in \mathbb{R}$ y $x = y$, entonces $x + z = y + z$ para cualquier $z$.
    \item Si $n, m \in \mathbb{Z}$, $n$ es impar y $n = m$, entonces $m$ es impar.
    \item Si $x = \sin(y)$ y $y = z^2$, por sustitución $\sin(y) = \sin(z^2)$, y por transitividad $x = \sin(z^2)$.
\end{itemize}
