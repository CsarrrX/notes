\section{Lógica}
Esta sección se basa en gran medida en el \textbf{Apéndice A} del libro \textit{Analysis I} de Terence Tao. Su objetivo es introducir los fundamentos de la \textbf{lógica matemática}, entendida como el lenguaje formal de las demostraciones. El estudio de esta lógica es indispensable, ya que en matemáticas nos interesa establecer verdades absolutas. Si bien existen otros sistemas lógicos que consideran nociones de verdad no absoluta, estos no son relevantes para los fines de las presentes notas.

\subsection{Enunciados matemáticos}
Todo argumento matemático está formado por \textit{enunciados matemáticos}. Estos son enunciados precisos que hacen referencia a distintos objetos matemáticos -como números, vectores o funciones- y a las relaciones u operaciones definidas entre ellos, tales como la suma, la integración, la igualdad o la composición. Dichos objetos pueden ser constantes o \textit{variables}. Un aspecto fundamental es que todo enunciado matemático posee un valor de verdad bien definido: \textbf{es verdadero o falso}, y no admite ambas posibilidades simultáneamente.

\begin{ejemplo}\label{ej1}
    $2 + 2 = 4$ es un enunciado \textit{verdadero} y $2 + 2 = 5$ es un enunciado \textit{falso}.
\end{ejemplo}

Cabe notar que no toda combinación de símbolos matemáticos es un enunciado matemático, por ejemplo la combinación: $= 2 + + 4 = + = 1$ \textbf{no} es un enunciado matemático y se le llama \textit{mal-formado} o \textit{mal-definido}. A los enunciados como \ref{ej1} se les llama \textit{bien-formados} o \textit{bien-definidos}. Dado esto, enunciados bien-definidos pueden ser o verdaderos o falsos, mientras que los mal-definidos no tienen ningún valor de verdad.

\begin{ejemplo}\label{ej2}
    $\frac{0}{0} = 1$ es un ejemplo de un enunciado mal-definido
\end{ejemplo}

Este ejemplo es un poco más sútil, sin embargo es mal-definido porque la división sobre 0 no esta definida. Un argumento matemático no debe contener ningún enunciado mal formado, entonces, si por ejemplo tenemos la expresion $\frac{x}{y} = z$ tenemos que asegurarnos que $y \neq 0$.
Siguiendo el tema del \textit{valor de verdad} de un enunciado, es importante señalar que, en la lógica matemática, la veracidad o falsedad de un enunciado es una propiedad inherente del mismo y no depende de la opinión de quien lo enuncia o interpreta. Esta característica junto con el valor de verdad bien definido de los enunciados fundamenta un método clásico de demostración: para establecer la veracidad de un enunciado, puede suponerse su falsedad y, a partir de dicha suposición, derivar una contradicción lógica. A este procedimiento se le denomina \textbf{demostración por contradicción}.

\begin{nota}
    Básicamente, para demostrar que un enunciado es verdadero, basta con demostrar que no es falso, y de la misma forma, para demostrar que falso, basta con demostrar que no es verdadero.
\end{nota}

Ahora, que un enunciado sea verdadero, no significa que sea útil o eficiente. Por ejemplo el enunciado $2 = 2$ es verdadero, pero esta lejos de ser algo útil. O el enunciado $4 \leq 4$ también es verdadero, pero no muy eficiente ($4 = 4$ es más preciso). Además, un enunciado puede ser falso pero muy útil, por ejemplo $\pi = 3.1416$ es falso, sin embargo es muy útil como una \textit{aproximación}. 

En razonamiento matemático, solo nos importa la \textbf{verdad}, no tanto la útilidad ni eficiencia; esto ya que la verdad es objetiva, podemos deducir enunciados verdaderos de reglas precisas, mientras que la útilidad y eficiencia subjetivas, dependen de la opinión y no siguen reglas precisas. 

Los enunciados son diferentes de las \textit{expresiones}. Como ya vimos los enunciados son verdaderos o falsos, mientras que las expresiones son \textit{secuencias} de caracteres matemáticos que producen un objeto matemático, por ejemplo, $2 \cdot 3 + 5$ es una expresión cuyo resultado es un número. Mientras que $ 2 \cdot 3 + 5 = 9$ es un enunciado, no una expresión. Como con los enunciados, las expresiones pueden ser bien o mal definidas, $2 + \frac{3}{0}$ es una expresión mal-definida. Ejemplos más sutiles de expresiones así ocurren, por ejemplo, cuando intentamos multiplicar matrices con dimensiones mal definidas, evaluamos funciones fuera de su dominio, etc. 

Podemos crear enunciados con expresiones al hacer uso de \textit{relaciones} como $=, <, \in$, etc. O propiedades (como: es primo, es continua, es invertible, etc.), por ejemplo $3 + 2$ es un número primo (los enunciados pueden contener palabras) o $3 + 28 < 40$ son enunciados matemáticos.
También podemos crear una \textit{proposición compuesta} usando enunciados más primitivos al \textit{conectarlos lógicamente}. 
% Fin de la subsección 

\subsection{Conectores lógicos} 

En lógica matemática, los conectores lógicos constituyen las herramientas fundamentales para construir y analizar proposiciones compuestas a partir de enunciados simples. Mediante estos operadores es posible formalizar relaciones como la conjunción, la disyunción, la implicación y la negación, permitiendo expresar razonamientos de manera precisa y rigurosa. A continuación se presentan los más utilizados. 

\begin{nota}
    Aunque en esencia un enunciado y una proposición son lo mismo, en estas notas nos referiremos a \textit{proposiciones} como enunciados un poco más complejos, sucesiones de enunciados más primitivos conectados lógicamente.
\end{nota}

\textbf{Conjunción.} Si $p$ es un enunciado y $q$ es un enunciado, entonces la proposición <<$p$ y $q$>> es verdadera si tanto $p$ como $q$ son verdaderos y falsa de otra forma. Por ejemplo, la proposición $2 + 2 = 4$ y $3 + 3 = 9$ es falsa, mientras que $2 + 2 = 4$ y $2 + 2 = 4$ es verdadera (aunque redundante).

\textbf{Disyunción.} Si $p$ es un enunciado y $q$ es un enunciado, entonces la proposición <<$p$ o $q$>> es verdadera si uno de los dos es verdadero (o ambos) y falsa solo cuando ambos son falsos. Por ejemplo, la proposición $2 + 2 = 1$ o $3 + 2 = 0$ es falsa, mientras que $2 = 2$ o $0 = 5$ es verdadera. 

\begin{nota}
    También existe el <<o exclusivo>>, en este solo nos interesa que UNO de los enunciados sea verdadero, si ambos son verdaderos entonces la proposición es falsa. 
\end{nota}

\textbf{Negación.} Si $p$ es un enunciado, la proposición <<$p$ no es verdadero>> es verdadera si y solo si $p$ es falso, y es falsa si y solo si $p$ es verdadero. La negación se presenta bastante en matemáticas (por ejemplo para demostrar por contradicción), gracias a esto ahora se presentan las reglas de la negación y ejemplos de negación de proposiciones.

Primero, la negación convierte la conjunción en disyunción y visceversa, por ejemplo, la proposición <<($p$ y $q$) es falsa>> se convierte en <<$p$ es falsa o $q$ es falsa>> (notese que también invierte el valor de verdad de los enunciados), y de la misma forma si tenemos <<($p$ o $q$) es falsa>> se convierte en <<$p$ es falsa y $q$ es falsa>>. Otra regla es que la doble negación se cancela, es decir, si tenemos un enunciado $p$ y la proposición <<$p$ es falso>>, la negación de esta proposición es <<$p$ es verdadero>>. 

\begin{ejemplo}
    Un ejemplo común de la negación se da en intervalos, por ejemplo, para negar $1 < x < 2$ tenemos que partir la proposición en $1 < x$ y $2 > x$ ahora, aplicando la regla para negarlo tenemos $1 \geq x$ o $2 \leq x$.
\end{ejemplo}

El dominio de la negación de proposiciones se da practicandola, no es algo particularmente complejo, sin embargo más adelante presentaremos cuantificadores y variables, los cuales conllevaran nuevas reglas de negación, entonces es importante recordar estas reglas básicas (también conocidas como las leyes de De Morgan).

\textbf{Si y solo si.} Si $p$ es un enunciado y $q$ es un enunciado, decimos <<$p$ es verdad si y solo si $q$ es verdad>> cuando son \textit{lógicamente equivalentes}, es decir, si $p$ es verdad $q$ es verdad, y si $p$ es falso entonces $q$ es falso, y de la misma forma, tenemos que si $q$ es verdadero, $p$ es verdadero y si $q$ es falso, entonces $p$ es falso.

\subsubsection{Implicación}

Hemos llegado a el conector lógico menos intuitivo, la \textbf{implicación}. Si $p$ es un enunciado y $q$ es un enunciado, entonces <<si $p$ entonces $q$>> es la implicación de $p$ a $q$. El significado de la sentencia <<si $p$, entonces $q$>> depende de si $p$ es verdadero o falso. Si $p$ es verdadero, entonces <<si $p$, entonces $q$>> es verdadero cuando $q$ es verdadero, y falso cuando $q$ es falso. Si, por el contrario, $p$ es falso, ¡entonces <<si $p$, entonces $q$>> es siempre verdadero, independientemente de si $q$ es verdadero o falso! Dicho de otra manera, cuando $p$ es verdadero, la sentencia <<si $p$, entonces $q$>> implica que $q$ es verdadero. Pero cuando $p$ es falso, la sentencia <<si $p$, entonces $q$>> no ofrece información sobre si $q$ es verdadero o no; la sentencia es verdadera, pero vacua (es decir, no transmite ninguna información nueva más allá del hecho de que la hipótesis es falsa).

\begin{nota}
    Una forma más intuitiva de entenderlo es con un ejemplo. Sean $p$: <<llueve>> y $q$: <<el suelo está mojado>>. La implicación <<si $p$, entonces $q$>> funciona como una regla o promesa:
    \begin{itemize}
        \item Si llueve ($p$ es V) y el suelo está mojado ($q$ es V), la regla se cumple: la implicación es \textbf{verdadera}.
        \item Si llueve ($p$ es V) pero el suelo está seco ($q$ es F), la regla se ha roto: la implicación es \textbf{falsa}.
        \item Si no llueve ($p$ es F), la regla no dice nada sobre el suelo. No importa si está mojado o seco, la afirmación original no ha sido contradicha, por lo que la implicación se considera \textbf{verdadera} por defecto.
    \end{itemize}
\end{nota}

Para demostrar una implicación, la manera usual de hacerlo es primero asumir que la hipótesis es verdadera y usar esto (junto con otros hechos e hipótesis verdaderas que conozcas) para deducir la conclusión. Este procedimiento es perfectamente válido, incluso si la hipótesis resulta falsa después; la implicación no garantiza nada acerca de la veracidad de la hipótesis, y solo garantiza la veracidad de la conclusión dado condicionalmente que la hipótesis sea verdadera. Por ejemplo, la demostración de la siguiente proposición es verdadera aunque tanto la hipótesis como la conclusión son falsas:


